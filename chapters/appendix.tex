% 附录
% 文件名:chapters/appendix.tex

\chapter{常用公式与算法}\label{app:formulas-algorithms}

本附录收集了数学建模中常用的公式、算法和工具,供读者参考。

\section{概率论与统计学公式}\label{app:probability-statistics}

\subsection{常用概率分布}

\begin{table}[htbp]
    \centering
    \caption{常用概率分布汇总}
    \label{tab:app-distributions}
    \begin{tabular}{@{}llll@{}}
        \toprule
        分布 & 参数 & 均值 & 方差 \\
        \midrule
        \multicolumn{4}{l}{\textbf{离散分布}} \\
        \midrule
        伯努利分布 & $p$ & $p$ & $p(1-p)$ \\
        二项分布 & $n, p$ & $np$ & $np(1-p)$ \\
        泊松分布 & $\lambda$ & $\lambda$ & $\lambda$ \\
        几何分布 & $p$ & $\frac{1}{p}$ & $\frac{1-p}{p^2}$ \\
        负二项分布 & $r, p$ & $\frac{r(1-p)}{p}$ & $\frac{r(1-p)}{p^2}$ \\
        \midrule
        \multicolumn{4}{l}{\textbf{连续分布}} \\
        \midrule
        均匀分布 & $a, b$ & $\frac{a+b}{2}$ & $\frac{(b-a)^2}{12}$ \\
        指数分布 & $\lambda$ & $\frac{1}{\lambda}$ & $\frac{1}{\lambda^2}$ \\
        正态分布 & $\mu, \sigma^2$ & $\mu$ & $\sigma^2$ \\
        伽马分布 & $\alpha, \beta$ & $\frac{\alpha}{\beta}$ & $\frac{\alpha}{\beta^2}$ \\
        贝塔分布 & $\alpha, \beta$ & $\frac{\alpha}{\alpha+\beta}$ & $\frac{\alpha\beta}{(\alpha+\beta)^2(\alpha+\beta+1)}$ \\
        \bottomrule
    \end{tabular}
\end{table}

\subsection{重要统计量的分布}

\begin{definitionbox}[title=抽样分布]
设 $X_1, X_2, \ldots, X_n$ 是来自 $\mathcal{N}(\mu, \sigma^2)$ 的样本,则:
\begin{enumerate}
    \item $\bar{X} \sim \mathcal{N}\left(\mu, \frac{\sigma^2}{n}\right)$
    \item $\frac{(n-1)S^2}{\sigma^2} \sim \chi^2(n-1)$
    \item $\frac{\bar{X} - \mu}{S/\sqrt{n}} \sim t(n-1)$
\end{enumerate}
\end{definitionbox}

\section{线性代数公式}\label{app:linear-algebra}

\subsection{矩阵基本运算}

\begin{infobox}[title=矩阵运算法则]
设 $A, B$ 是矩阵,$c$ 是标量,则:
\begin{itemize}
    \item $(A + B)^T = A^T + B^T$
    \item $(AB)^T = B^T A^T$
    \item $(A^{-1})^T = (A^T)^{-1}$
    \item $\det(AB) = \det(A)\det(B)$
    \item $\text{tr}(A + B) = \text{tr}(A) + \text{tr}(B)$
    \item $\text{tr}(AB) = \text{tr}(BA)$
\end{itemize}
\end{infobox}

\subsection{特殊矩阵}

\begin{table}[htbp]
    \centering
    \caption{特殊矩阵类型}
    \label{tab:special-matrices}
    \begin{tabular}{@{}ll@{}}
        \toprule
        矩阵类型 & 性质 \\
        \midrule
        对称矩阵 & $A^T = A$,特征值为实数 \\
        反对称矩阵 & $A^T = -A$,对角元素为零 \\
        正交矩阵 & $A^T A = I$,$\det(A) = \pm 1$ \\
        幂等矩阵 & $A^2 = A$,特征值为 0 或 1 \\
        正定矩阵 & $x^T A x > 0$ 对所有 $x \neq 0$ \\
        \bottomrule
    \end{tabular}
\end{table}

\section{优化算法}\label{app:optimization-algorithms}

\subsection{梯度下降法}

\begin{algorithm}[H]
\caption{梯度下降法}\label{alg:gradient-descent}
\begin{algorithmic}[1]
\State 初始化 $x_0$,设置学习率 $\alpha > 0$,容差 $\epsilon > 0$
\State $k = 0$
\While{$\|\nabla f(x_k)\| > \epsilon$}
    \State $x_{k+1} = x_k - \alpha \nabla f(x_k)$
    \State $k = k + 1$
\EndWhile
\State 返回 $x_k$
\end{algorithmic}
\end{algorithm}

\begin{codebox}[title=梯度下降法的 Python 实现]
\begin{minted}{python}
import numpy as np

def gradient_descent(f, grad_f, x0, alpha=0.01, epsilon=1e-6, max_iter=1000):
    """
    梯度下降法求解无约束优化问题
    
    参数:
    f: 目标函数
    grad_f: 梯度函数
    x0: 初始点
    alpha: 学习率
    epsilon: 收敛容差
    max_iter: 最大迭代次数
    """
    x = x0.copy()
    history = [x.copy()]
    
    for i in range(max_iter):
        grad = grad_f(x)
        
        # 检查收敛性
        if np.linalg.norm(grad) < epsilon:
            break
        
        # 更新参数
        x = x - alpha * grad
        history.append(x.copy())
    
    return x, history

# 示例:最小化 f(x) = x1^2 + x2^2
def f(x):
    return x[0]**2 + x[1]**2

def grad_f(x):
    return np.array([2*x[0], 2*x[1]])

# 求解
x0 = np.array([1.0, 1.0])
x_opt, history = gradient_descent(f, grad_f, x0)
print(f"最优解: {x_opt}")
print(f"最优值: {f(x_opt)}")
\end{minted}
\end{codebox}

\subsection{牛顿法}

\begin{algorithm}[H]
\caption{牛顿法}\label{alg:newton-method}
\begin{algorithmic}[1]
\State 初始化 $x_0$,设置容差 $\epsilon > 0$
\State $k = 0$
\While{$\|\nabla f(x_k)\| > \epsilon$}
    \State 计算 Hessian 矩阵 $H_k = \nabla^2 f(x_k)$
    \State 解方程 $H_k d_k = -\nabla f(x_k)$
    \State $x_{k+1} = x_k + d_k$
    \State $k = k + 1$
\EndWhile
\State 返回 $x_k$
\end{algorithmic}
\end{algorithm}

\section{数值计算方法}\label{app:numerical-methods}

\subsection{数值积分}

\begin{table}[htbp]
    \centering
    \caption{数值积分公式}
    \label{tab:numerical-integration}
    \begin{tabular}{@{}ll@{}}
        \toprule
        方法 & 公式 \\
        \midrule
        梯形公式 & $\int_a^b f(x) dx \approx \frac{b-a}{2}[f(a) + f(b)]$ \\
        Simpson公式 & $\int_a^b f(x) dx \approx \frac{b-a}{6}[f(a) + 4f(\frac{a+b}{2}) + f(b)]$ \\
        复化梯形公式 & $\int_a^b f(x) dx \approx \frac{h}{2}[f(a) + 2\sum_{i=1}^{n-1}f(x_i) + f(b)]$ \\
        \bottomrule
    \end{tabular}
\end{table}

\subsection{常微分方程数值解}

\begin{examplebox}[title=欧拉方法]
对于初值问题 $y' = f(t, y)$,$y(t_0) = y_0$,欧拉方法的递推公式为:
\[
y_{n+1} = y_n + h f(t_n, y_n)
\]
其中 $h$ 是步长,$t_n = t_0 + nh$。
\end{examplebox}

\begin{codebox}[title=四阶龙格-库塔方法]
\begin{minted}{python}
import numpy as np

def runge_kutta_4(f, t0, y0, h, n):
    """
    四阶龙格-库塔方法求解常微分方程
    
    参数:
    f: 函数 f(t, y)
    t0: 初始时间
    y0: 初始值
    h: 步长
    n: 步数
    """
    t = np.zeros(n+1)
    y = np.zeros(n+1)
    
    t[0] = t0
    y[0] = y0
    
    for i in range(n):
        k1 = h * f(t[i], y[i])
        k2 = h * f(t[i] + h/2, y[i] + k1/2)
        k3 = h * f(t[i] + h/2, y[i] + k2/2)
        k4 = h * f(t[i] + h, y[i] + k3)
        
        y[i+1] = y[i] + (k1 + 2*k2 + 2*k3 + k4) / 6
        t[i+1] = t[i] + h
    
    return t, y

# 示例:求解 y' = -2y + 1, y(0) = 0
def f(t, y):
    return -2*y + 1

t, y = runge_kutta_4(f, 0, 0, 0.1, 100)
\end{minted}
\end{codebox}

\section{图论算法}\label{app:graph-algorithms}

\subsection{最短路径算法}

\begin{algorithm}[H]
\caption{Dijkstra 算法}\label{alg:dijkstra}
\begin{algorithmic}[1]
\State 初始化距离数组 $d$,$d[s] = 0$,其余为 $\infty$
\State 初始化优先队列 $Q$,包含所有顶点
\While{$Q$ 非空}
    \State $u = \arg\min_{v \in Q} d[v]$
    \State 从 $Q$ 中移除 $u$
    \For{$u$ 的每个邻居 $v$}
        \State $alt = d[u] + w(u, v)$
        \If{$alt < d[v]$}
            \State $d[v] = alt$
            \State $prev[v] = u$
        \EndIf
    \EndFor
\EndWhile
\end{algorithmic}
\end{algorithm}

\subsection{最小生成树算法}

\begin{algorithm}[H]
\caption{Prim 算法}\label{alg:prim}
\begin{algorithmic}[1]
\State 选择任意顶点 $v_0$ 作为起始点
\State 初始化 $V_{MST} = \{v_0\}$,$E_{MST} = \emptyset$
\While{$|V_{MST}| < |V|$}
    \State 找到权重最小的边 $(u, v)$,其中 $u \in V_{MST}$,$v \notin V_{MST}$
    \State $V_{MST} = V_{MST} \cup \{v\}$
    \State $E_{MST} = E_{MST} \cup \{(u, v)\}$
\EndWhile
\end{algorithmic}
\end{algorithm}

\section{常用软件工具}\label{app:software-tools}

\subsection{Python 科学计算库}

\begin{table}[htbp]
    \centering
    \caption{Python 科学计算库}
    \label{tab:python-libraries}
    \begin{tabular}{@{}ll@{}}
        \toprule
        库名 & 主要功能 \\
        \midrule
        NumPy & 数值计算、数组操作 \\
        SciPy & 科学计算、优化、统计 \\
        Matplotlib & 数据可视化 \\
        Pandas & 数据处理和分析 \\
        Scikit-learn & 机器学习 \\
        SymPy & 符号计算 \\
        NetworkX & 图论和网络分析 \\
        CVXPY & 凸优化 \\
        \bottomrule
    \end{tabular}
\end{table}

\subsection{MATLAB 工具箱}

\begin{table}[htbp]
    \centering
    \caption{MATLAB 常用工具箱}
    \label{tab:matlab-toolboxes}
    \begin{tabular}{@{}ll@{}}
        \toprule
        工具箱 & 功能 \\
        \midrule
        Optimization Toolbox & 优化算法 \\
        Statistics Toolbox & 统计分析 \\
        Signal Processing Toolbox & 信号处理 \\
        Control System Toolbox & 控制系统 \\
        Curve Fitting Toolbox & 曲线拟合 \\
        Partial Differential Equation Toolbox & 偏微分方程 \\
        Global Optimization Toolbox & 全局优化 \\
        \bottomrule
    \end{tabular}
\end{table}

\section{数学建模竞赛常用技巧}\label{app:modeling-tips}

\subsection{问题分析技巧}

\begin{successbox}[title=问题分析的一般步骤]
\begin{enumerate}
    \item \textbf{理解问题背景}:深入理解问题的实际背景和应用场景
    \item \textbf{明确目标}:清晰地定义要解决的问题和期望的结果
    \item \textbf{识别变量}:确定决策变量、参数和约束条件
    \item \textbf{简化假设}:合理简化复杂的实际问题
    \item \textbf{选择方法}:根据问题特点选择合适的数学方法
    \item \textbf{验证结果}:检验模型的合理性和准确性
\end{enumerate}
\end{successbox}

\subsection{模型建立技巧}

\begin{warningbox}[title=模型建立的注意事项]
\begin{itemize}
    \item 避免过度复杂化,保持模型的简洁性
    \item 注意模型的适用范围和局限性
    \item 考虑模型的可解性和计算复杂度
    \item 验证模型的稳定性和敏感性
    \item 对结果进行合理性检验
\end{itemize}
\end{warningbox}

\subsection{论文写作技巧}

\begin{infobox}[title=数学建模论文结构]
\begin{enumerate}
    \item \textbf{摘要}:简洁地概述问题、方法和结果
    \item \textbf{问题重述}:用自己的话重新表述问题
    \item \textbf{模型假设}:列出所有的假设条件
    \item \textbf{符号说明}:定义所有使用的符号
    \item \textbf{模型建立}:详细描述模型的建立过程
    \item \textbf{模型求解}:说明求解方法和计算过程
    \item \textbf{结果分析}:分析和解释计算结果
    \item \textbf{模型评价}:讨论模型的优缺点
    \item \textbf{参考文献}:列出所有引用的资料
\end{enumerate}
\end{infobox}

\section{常用数学公式}\label{app:mathematical-formulas}

\subsection{微积分公式}

\begin{table}[htbp]
    \centering
    \caption{常用导数公式}
    \label{tab:derivatives}
    \begin{tabular}{@{}ll@{}}
        \toprule
        函数 & 导数 \\
        \midrule
        $x^n$ & $nx^{n-1}$ \\
        $e^x$ & $e^x$ \\
        $\ln x$ & $\frac{1}{x}$ \\
        $\sin x$ & $\cos x$ \\
        $\cos x$ & $-\sin x$ \\
        $\tan x$ & $\sec^2 x$ \\
        $\arcsin x$ & $\frac{1}{\sqrt{1-x^2}}$ \\
        $\arccos x$ & $-\frac{1}{\sqrt{1-x^2}}$ \\
        $\arctan x$ & $\frac{1}{1+x^2}$ \\
        \bottomrule
    \end{tabular}
\end{table}

\subsection{积分公式}

\begin{table}[htbp]
    \centering
    \caption{常用积分公式}
    \label{tab:integrals}
    \begin{tabular}{@{}ll@{}}
        \toprule
        函数 & 积分 \\
        \midrule
        $x^n$ & $\frac{x^{n+1}}{n+1} + C$ \\
        $e^x$ & $e^x + C$ \\
        $\frac{1}{x}$ & $\ln|x| + C$ \\
        $\sin x$ & $-\cos x + C$ \\
        $\cos x$ & $\sin x + C$ \\
        $\sec^2 x$ & $\tan x + C$ \\
        $\frac{1}{\sqrt{1-x^2}}$ & $\arcsin x + C$ \\
        $\frac{1}{1+x^2}$ & $\arctan x + C$ \\
        \bottomrule
    \end{tabular}
\end{table}

\subsection{级数公式}

\begin{definitionbox}[title=常用级数展开]
\begin{align}
e^x &= \sum_{n=0}^{\infty} \frac{x^n}{n!} = 1 + x + \frac{x^2}{2!} + \frac{x^3}{3!} + \cdots \\
\sin x &= \sum_{n=0}^{\infty} \frac{(-1)^n x^{2n+1}}{(2n+1)!} = x - \frac{x^3}{3!} + \frac{x^5}{5!} - \cdots \\
\cos x &= \sum_{n=0}^{\infty} \frac{(-1)^n x^{2n}}{(2n)!} = 1 - \frac{x^2}{2!} + \frac{x^4}{4!} - \cdots \\
\ln(1+x) &= \sum_{n=1}^{\infty} \frac{(-1)^{n+1} x^n}{n} = x - \frac{x^2}{2} + \frac{x^3}{3} - \cdots
\end{align}
\end{definitionbox}

\section{编程实用技巧}\label{app:programming-tips}

\subsection{Python 编程技巧}

\begin{codebox}[title=Python 常用技巧]
\begin{minted}{python}
# 1. 列表推导式
squares = [x**2 for x in range(10)]

# 2. 字典推导式
square_dict = {x: x**2 for x in range(5)}

# 3. 函数装饰器
def timer(func):
    import time
    def wrapper(*args, **kwargs):
        start = time.time()
        result = func(*args, **kwargs)
        end = time.time()
        print(f"{func.__name__} took {end - start:.4f} seconds")
        return result
    return wrapper

@timer
def slow_function():
    time.sleep(1)
    return "Done"

# 4. 上下文管理器
with open('file.txt', 'r') as f:
    content = f.read()

# 5. 异常处理
try:
    result = 10 / 0
except ZeroDivisionError as e:
    print(f"Error: {e}")
finally:
    print("Cleanup code")
\end{minted}
\end{codebox}

\subsection{数据处理技巧}

\begin{codebox}[title=Pandas 常用操作]
\begin{minted}{python}
import pandas as pd
import numpy as np

# 创建数据框
df = pd.DataFrame({
    'A': [1, 2, 3, 4, 5],
    'B': [10, 20, 30, 40, 50],
    'C': ['a', 'b', 'c', 'd', 'e']
})

# 数据筛选
filtered_df = df[df['A'] > 2]

# 数据分组
grouped = df.groupby('C')['A'].mean()

# 数据透视
pivot_table = df.pivot_table(values='B', index='C', columns='A')

# 缺失值处理
df_filled = df.fillna(method='forward')

# 数据合并
df1 = pd.DataFrame({'key': ['A', 'B'], 'value1': [1, 2]})
df2 = pd.DataFrame({'key': ['A', 'B'], 'value2': [3, 4]})
merged = pd.merge(df1, df2, on='key')
 \end{minted}
\end{codebox}

\section{参考资料}\label{app:references}

\subsection{推荐书籍}

\begin{enumerate}
    \item 《数学建模》- 姜启源等
    \item 《数学建模算法与应用》- 司守奎等
    \item 《Mathematical Modeling》- Frank R. Giordano
    \item 《A First Course in Mathematical Modeling》- Frank R. Giordano
    \item 《Optimization Theory and Methods》- Wenyu Sun
\end{enumerate}

\subsection{在线资源}

\begin{itemize}
    \item 全国大学生数学建模竞赛官网:\url{http://www.mcm.edu.cn}
    \item 美国大学生数学建模竞赛官网:\url{https://www.comap.com/contests/mcm-icm}
    \item SciPy 官方文档:\url{https://scipy.org}
    \item NumPy 官方文档:\url{https://numpy.org}
    \item Matplotlib 官方文档:\url{https://matplotlib.org}
\end{itemize}

\subsection{数学建模竞赛}

\begin{table}[htbp]
    \centering
    \caption{主要数学建模竞赛}
    \label{tab:modeling-competitions}
    \begin{tabular}{@{}ll@{}}
        \toprule
        竞赛名称 & 举办时间 \\
        \midrule
        全国大学生数学建模竞赛 & 每年9月 \\
        美国大学生数学建模竞赛 & 每年2月 \\
        全国研究生数学建模竞赛 & 每年9月 \\
        "高教社杯"全国大学生数学建模竞赛 & 每年9月 \\
                 \bottomrule
     \end{tabular}
 \end{table}
 
 这些竞赛为学生提供了实践数学建模技能的机会,也是检验学习成果的重要平台。 